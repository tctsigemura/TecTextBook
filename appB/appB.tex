\renewcommand{\myincludegraphics}[2]{\includegraphics[#2]{appB/#1}}

\newpage
%\onecolumn
\chapter{TeCクロス開発環境}
\label{cross}
\section{始めに}
TeCクロス開発環境は,
教育用コンピュータTeC(Tokuyama kousen Educational Computer)の
機械語プログラムを開発するためのパソコン上で動作するプログラム群です.
クロス開発環境は,クロスアセンブラtasm7,
ダウンロードプログラムtwrite7からなります.
これらを用いてTeCのプログラムをパソコン上でクロス開発する
様子を\figref{appB:cross}に示します.

これらの入手とパソコンへのインストール方法は,
\url{https://github.com/tctsigemura/Tasm}を参照して下さい.
以下では,
UNIXまたはmacOSにインストールされた「TeCクロス開発環境」を
使用する前提で説明しています.

\myfigureNA{btp}{scale=0.75}{develop-crop.pdf}{処理の流れ}{appB:cross}

\section{アセンブルコマンド}
クロスアセンブラは次の形式のコマンド行から起動され,
ソースプログラムを入力し,アセンブルリストと機械語を出力します.
アセンブルリストは拡張子 .lst のファイルに,
機械語は拡張子 .bin のファイルに格納されます.

\begin{center}
{\small\tt tasm7 [-l nn] [-w nn] {\it source\_file} }
\end{center}

{\it source\_file}は\ref{syn}章で説明する文法で記述された
ソースファイルの名前です.
ソースファイル名は,拡張子 .t7 でなければなりません.
アセンブルリストの整形に関するコマンド行オプションが使用できます.
意味は\tabref{appB:option}の通りです.

\begin{mytable}{tbp}{コマンド行オプション}{appB:option}
{\small\begin{tabular}{l | l} \hline\hline
\multicolumn{1}{c|}{\it オプション} & \multicolumn{1}{c}{\it 意味} \\ \hline
{\tt -l nn}  &  リストの1ページをnn行にする. \\
{\tt -w nn}  &  リストの1ページをnn桁にする.
\end{tabular}}
\end{mytable}

\section{ダウンロードプログラム}
ダウンロードプログラムはtasm7が出力した機械語プログラムを
シリアル通信を用いてTeCに転送します.
次の形式のコマンド行から起動します.

\begin{center}
{\small\tt twrite7  {\it bin\_file}} \\
\end{center}

{\it bin\_file}はtasm7が出力した機械語プログラムファイルの名前です.
twrite7を起動すると,
{\tt .bin}ファイルがTeCの主記憶装置(メモリ)に自動的に書き込まれます.

\section{アセンブラの文法}
\label{syn}
以下ではtasm7の入力となるソースプログラムの文法について説明します.

\subsection{行フォーマット}
ソースプログラムは「行」の連なりからなるテキストです.
ソースプログラムはソースファイルと呼ばれるテキストファイルに格納されます.

tasm7のソースプログラムでは,文字定数と文字列中を除き,
英字大文字と小文字は区別されません.
空白はトークンの前後にいくつでも挿入できます.
但し,行頭の空白だけは「ラベルが存在しない」ことを表す特別なものになります.
空白にはスペース(文字コード20H)とタブ(文字コード09H)の2種類があります.
以下に「行」の書式を示します.

{\small\[ [ラベル]~~[命令とオペランド]~~[ ; コメント] \]}

\subsection{ラベル}
行が英字で始まる場合,その行にラベルがあるとみなされます.
ラベルは英字で始まり,その後に任意長の英数字が続く文字列です.
ラベルが無い行は空白文字で始めます.
ラベルの長さに制限はありませんが,
アセンブルリストの見栄えから7文字以内を推奨します.

行にラベルを書くとラベルが定義されます.
EQU命令を除き,その行のアドレスがラベルの値になります.

\subsection{疑似命令}
以下の4種類の疑似命令が使用できます.

\subsubsection{EQU命令}
任意の値を持つラベルを定義する疑似命令です.
この命令にはラベルが必ず必要です.
命令行のラベルがオペランド式の値で定義されます.
式ではラベルの前方参照はできません.
EQU命令の書式を以下に,使用例を\figref{appB:equ}に示します.

{\small\[ ラベル~~EQU~~式 \]}

\begin{figure}[btp]
\begin{center}
{\tt\small\begin{tabular}{|l|l|l|}\hline
\multicolumn{1}{|c|}{ラベル} & 
        \multicolumn{1}{c|}{命令} & \multicolumn{1}{c|}{オペランド} \\\hline
START & EQU & 03H \\
SIZE  & EQU & 100 \\
END   & EQU & START+SIZE \\\hline
\end{tabular}}
\caption{EQU 命令の使用例}
\label{fig:appB:equ}
\end{center}
\end{figure}

\subsubsection{ORG命令}
機械語を生成するアドレスを変更する疑似命令です.
機械語(データも含む)が生成される全ての行より前で使用された場合は,
単に次の命令のアドレスを変更します.
機械語の生成が開始された後に使用された場合は,
指定アドレスに達するまでゼロで初期化された領域を生成します.

現在のアドレスより若いアドレス(逆戻りするアドレス)を
指定することはできません.
ORG命令の式ではラベルの前方参照は禁止です.
ORG命令の書式を以下に,使用例を\figref{appB:org}に示します.

{\small\[ [ラベル]~~ORG~~式\]}

\begin{figure}[btp]
\begin{center}
{\tt\small\begin{tabular}{|l|l|l|}\hline
\multicolumn{1}{|c|}{ラベル} & 
        \multicolumn{1}{c|}{命令} & \multicolumn{1}{c|}{オペランド} \\\hline
START & ORG & 03H \\\hline
\end{tabular}}
\caption{ORG 命令の使用例}
\label{fig:appB:org}
\end{center}
\end{figure}

\subsubsection{DS命令}
領域を定義するために使用する疑似命令です.
式の値は領域のバイト数を表します.
領域の値はゼロで初期化されます.
DS命令の式では,ラベルの前方参照はできません.
DS命令の書式を以下に,使用例を\figref{appB:ds}に示します.

{\small\[ [ラベル]~~DS~~式 \]}

\begin{figure}[btp]
\begin{center}
{\tt\small\begin{tabular}{|l|l|l|}\hline
\multicolumn{1}{|c|}{ラベル} & 
        \multicolumn{1}{c|}{命令} & \multicolumn{1}{c|}{オペランド} \\\hline
WORK & DS & 10 \\\hline
\end{tabular}}
\caption{DS 命令の使用例}
\label{fig:appB:ds}
\end{center}
\end{figure}

\subsubsection{DC命令}
メモリ上に任意の定数を出力するために使用する疑似命令です.
オペランドに任意個の式または文字列を書くことができます.
DC命令の書式を以下に,使用例を\figref{appB:dc}に示します.

{\small\[ %DC命令 =
[ラベル]~~DC~~\left\{
  \begin{array}{c}
   式 \\
   文字列
  \end{array}
 \right\}
[,\left\{
  \begin{array}{c}
   式 \\
   文字列
  \end{array}  
 \right\} ~ ... ~
]
\]}

\begin{figure}[btp]
\begin{center}
{\tt\small\begin{tabular}{|l|l|l|}\hline
\multicolumn{1}{|c|}{ラベル} & 
        \multicolumn{1}{c|}{命令} & \multicolumn{1}{c|}{オペランド} \\\hline
DATA & DC & 1,2,3,4 \\
     & DC & 1+1 \\
CHA  & DC & 'a' \\
STR  & DC & "abc" \\\hline
\end{tabular}}
\caption{DC 命令の使用例}
\label{fig:appB:dc}
\end{center}
\end{figure}

\subsection{機械語命令}
機械語命令は6つのグループに分類できます.
以下では6つのグループの命令について,書き方を説明します.

\subsubsection{機械語命令1}
オペランドの無い1バイトの機械語命令です.
\tabref{appB:m1}の命令がこのグループに属します.
このグループの命令の書式を以下に,使用例を\figref{appB:m1ex}に示します.

{\small\[ %機械語命令1 =
[ラベル]~~命令 \]}

\begin{mytable}{btp}{機械語命令1に属す命令}{appB:m1}
{\small\begin{tabular}{l|l}
\hline\hline
\multicolumn{1}{c|}{命令} & \multicolumn{1}{c}{意味} \\\hline
NO & 何もしない \\
EI & 割り込み許可 \\
DI & 割り込み禁止 \\
RET & サブルーチンから復帰 \\
RETI & 割り込みから復帰 \\
HALT & プログラム終了
\end{tabular}}
\end{mytable}

\begin{figure}[btp]
\begin{center}
{\tt\small\begin{tabular}{|l|l|l|}
\hline
\multicolumn{1}{|c|}{ラベル} & 
        \multicolumn{1}{c|}{命令} & \multicolumn{1}{c|}{オペランド} \\\hline
START & NO   &  \\
      & NO   &  \\
END   & HALT &  \\\hline
\end{tabular}}
\caption{機械語命令1の使用例}
\label{fig:appB:m1ex}
\end{center}
\end{figure}

\subsubsection{機械語命令2}
レジスタ指定付きの1バイトの機械語命令です.
レジスタとしてG0,G1,G2,SPが使用できます.
\tabref{appB:m2}の命令がこのグループに属します.
このグループの命令の書式を以下に,使用例を\figref{appB:m2ex}に示します.

{\small\[ % 機械語命令2 =
[ラベル]~~命令~~レジスタ \]}

\begin{mytable}{btp}{機械語命令2に属す命令}{appB:m2}
{\small\begin{tabular}{l|l}
\hline\hline
\multicolumn{1}{c|}{命令} & \multicolumn{1}{c}{意味} \\\hline
SHLA & 左算術シフト \\
SHLL & 左論理シフト \\
SHRA & 右算術シフト \\
SHRL & 右論理シフト \\
PUSH & スタックにレジスタを退避 \\
POP  & スタックからレジスタを復旧
\end{tabular}}
\end{mytable}

\begin{figure}[btp]
\begin{center}
{\tt\small\begin{tabular}{|l|l|l|}\hline
\multicolumn{1}{|c|}{ラベル} & 
        \multicolumn{1}{c|}{命令} & \multicolumn{1}{c|}{オペランド} \\\hline
L1  & SHRA &  SP \\
    & PUSH &  G0 \\
    & POP  &  G1 \\\hline
\end{tabular}}
\caption{機械語命令2の使用例}
\label{fig:appB:m2ex}
\end{center}
\end{figure}

\subsubsection{機械語命令3}
オペランドにレジスタと式を書く2バイトの機械語命令です.
レジスタとしてG0,G1,G2,SPが使用できます.
\tabref{appB:m3}の命令がこのグループに属します.
このグループの命令の書式を以下に,使用例を\figref{appB:m3ex}に示します.

{\small\[ % 機械語命令3 =
[ラベル]~~命令~~レジスタ,式 \]}

\begin{mytable}{btp}{機械語命令3に属す命令}{appB:m3}
{\small\begin{tabular}{l|l}
\hline\hline
\multicolumn{1}{c|}{命令} & \multicolumn{1}{c}{意味} \\\hline
IN & 入出力ポートから読み込む \\
OUT & 入出力ポートに書き込む
\end{tabular}}
\end{mytable}

\begin{figure}[btp]
\begin{center}
{\tt\small\begin{tabular}{|l|l|l|}\hline
\multicolumn{1}{|c|}{ラベル} & 
        \multicolumn{1}{c|}{命令} & \multicolumn{1}{c|}{オペランド} \\\hline
SIO  & EQU &  03H \\
     & IN  &  G0,SIO \\\hline
\end{tabular}}
\caption{機械語命令3の使用例}
\label{fig:appB:m3ex}
\end{center}
\end{figure}

\subsubsection{機械語命令4}
レジスタとメモリをオペランドに指定できる命令で,
全てのアドレッシング・モードが使用できるグループです.
レジスタとしてG0,G1,G2,SPが,
インデクスレジスタとしてG1,G2が使用できます.
\tabref{appB:m4}の命令がこのグループに属します.
このグループの命令の書式を以下に,使用例を\figref{appB:m4ex}に示します.

オペランドの書式はアドレッシング・モードにより変化します.
1行目がダイレクトモード,2行目がインデクスドモード,
3行目がイミディエイトモードです.

{\small\[ %機械語命令4 = 
[ラベル]~~命令~~\left\{
  \begin{array}{c}
   レジスタ,式 \\
   レジスタ,式,インデクスレジスタ \\
   レジスタ,#式 \\
  \end{array}  
 \right\}
 \]}

\begin{mytable}{btp}{機械語命令4に属す命令}{appB:m4}
{\small\begin{tabular}{l|l}
\hline\hline
\multicolumn{1}{c|}{命令} & \multicolumn{1}{c}{意味} \\\hline
LD & レジスタにメモリの値をロード \\
ADD & レジスタとメモリの値の和 \\
SUB & レジスタとメモリの値の差 \\
CMP & レジスタとメモリの値を比較 \\
AND & レジスタとメモリの値の論理積 \\
OR & レジスタとメモリの値の論理和 \\
XOR & レジスタとメモリの値の排他的論理和
\end{tabular}}
\end{mytable}

\begin{figure}[btp]
\begin{center}
{\tt\small\begin{tabular}{|l|l|l|}\hline
\multicolumn{1}{|c|}{ラベル} & 
        \multicolumn{1}{c|}{命令} & \multicolumn{1}{c|}{オペランド} \\\hline
L1  & LD  & G0,DATA \\
    & LD  & G1,\#0 \\
    & SUB & G0,DATA,G1 \\
    & ADD & G1,\#1 \\\hline
\end{tabular}}
\caption{機械語命令4の使用例}
\label{fig:appB:m4ex}
\end{center}
\end{figure}

\subsubsection{機械語命令5}
機械語命令4と,ほぼ,同様ですが,
イミディエイトモードが使用できない機械語命令のグループです.
レジスタとしてG0,G1,G2,SPが,
インデクスレジスタとしてG1,G2が使用できます.
\tabref{appB:m5}のようにST命令だけがこのグループに属します.
このグループの命令の書式を以下に,使用例を\figref{appB:m5ex}に示します.

{\small\[ %機械語命令5 = 
[ラベル]~~命令~~\left\{
  \begin{array}{c}
   レジスタ,式 \\
   レジスタ,式,インデクスレジスタ \\
  \end{array}  
 \right\}
 \]}

\begin{mytable}{btp}{機械語命令5に属す命令}{appB:m5}
{\small\begin{tabular}{l|l}
\hline\hline
\multicolumn{1}{c|}{命令} & \multicolumn{1}{c}{意味} \\\hline
ST & レジスタの値をメモリオペランドに格納
\end{tabular}}
\end{mytable}

\begin{figure}[btp]
\begin{center}
{\tt\small\begin{tabular}{|l|l|l|}\hline
\multicolumn{1}{|c|}{ラベル} & 
        \multicolumn{1}{c|}{命令} & \multicolumn{1}{c|}{オペランド} \\\hline
  & ST & G0,WORK \\
  & ST & G0,WORK,G1 \\\hline
\end{tabular}}
\caption{機械語命令5の使用例}
\label{fig:appB:m5ex}
\end{center}
\end{figure}

\subsubsection{機械語命令6}
メモリオペランドだけの機械語命令のグループです.
\tabref{appB:m6}に示すジャンプ・コール命令が,このグループに属します.
インデクスレジスタとしてG1,G2が使用できます.
このグループの命令の書式を以下に,使用例を\figref{appB:m6ex}に示します.

{\small\[ %機械語命令6 =
[ラベル]~~命令~~\left\{
  \begin{array}{c}
   式 \\
   式,インデクスレジスタ \\
  \end{array}  
 \right\}
 \]}

\begin{mytable}{btp}{機械語命令6に属す命令}{appB:m6}
{\small\begin{tabular}{l|l}
\hline\hline
\multicolumn{1}{c|}{\it 命令} & \multicolumn{1}{c}{\it 意味} \\
\hline
JMP & 必ずジャンプ \\
JZ  & Zフラグが'1'ならジャンプ \\
JC  & Cフラグに'1'ならジャンプ \\
JM  & Sフラグに'1'ならジャンプ \\
JNZ & Zフラグが'0'ならジャンプ \\
JNC & Cフラグに'0'ならジャンプ \\
JNM & Sフラグに'0'ならジャンプ \\
CALL & サブルーチン呼び出し
\end{tabular}}
\end{mytable}

\begin{figure}[btp]
\begin{center}
{\tt\small\begin{tabular}{|l|l|l|}
\hline
\hline
\multicolumn{1}{|c|}{ラベル} & 
        \multicolumn{1}{c|}{命令} & \multicolumn{1}{c|}{オペランド} \\
  & JMP & LOOP \\
  & CALL & SUB1 \\
\hline
\end{tabular}}
\caption{機械語命令6の使用例}
\label{fig:appB:m6ex}
\end{center}
\end{figure}

\section{アセンブラの文法まとめ}
次のページにアセンブラの文法を BNF 風にまとめます.

\onecolumn

%\begin{figure*}[btp]
\begin{center}
{\small\tt\begin{tabular}{lll}
<プログラム>& ::= &{ <行> } EOF \\
<行>        & ::= &<命令行> | <空行> | <コメント行> \\
<命令行>    & ::= &<ラベル欄> [<命令欄>] [<コメント>] <改行> \\
<空行>      & ::= &<改行> \\
<コメント行>& ::= &<コメント> <改行> \\
<ラベル欄>  & ::= &<ラベル> | <空白> \\
<命令欄>    & ::= &<命令記述1> | <命令記述2> | <命令記述3>
                       | <命令記述4> \\
              &     &  | <命令記述5> | <命令記述6> 
                       | <命令記述7> | <命令記述8> \\
<命令記述1>& ::= &<命令1> \\
<命令記述2>& ::= &<命令2> <レジスタ> \\
<命令記述3>& ::= &<命令3> <レジスタ> , <式> \\
<命令記述4>& ::= &<命令4> <レジスタ> , <アドレス1> \\
<命令記述5>& ::= &<命令5> <レジスタ> , <アドレス2> \\
<命令記述6>& ::= &<命令6> <アドレス2> \\
<命令記述7>& ::= &<命令7> <式> \\
<命令記述8>& ::= &<命令8> <値列> \\
<アドレス1>& ::= &\#<式> | <式> [,<インデクス>] \\
<アドレス2>& ::= &<式> [,<インデクス>] \\
<値列>      & ::= &<拡張式> { ,<拡張式> } \\
<拡張式>    & ::= &<式> | <文字列> \\
<式>        & ::= &<項> { ( +|- ) <項> } \\
<項>        & ::= &<因子1> { ( *|/ ) <因子1> } \\
<因子1>    & ::= &<因子> | +<因子> | -<因子> \\
<因子>      & ::= &<数値> | <ラベル> | ( <式> ) \\
<コメント>  & ::= &; { <文字1> } \\
<数値>      & ::= &<10進数> | <16進数> | <文字定数> \\
<10進数>  & ::= &<数字> { <数字> } \\
<16進数>  & ::= &<数字> { <16進数字> } H \\
<文字定数>  & ::= &'<文字2>' \\
<文字列>    & ::= &"{ <文字3> }" \\
<ラベル>    & ::= &<英字> { <英数字> } \\
<英数字>    & ::= &<英字> | <数字> \\
<文字1>    & ::= &<改行>以外の文字 \\
<文字2>    & ::= &<改行>,' 以外の文字 \\
<文字3>    & ::= &<改行>,” 以外の文字 \\
<16進数字>& ::= &0 | 1 | 2 | 3 | 4 | 5 | 6 | 7 | 8 | 9 
                       | A | B | C | D | E | F \\
<英字>      & ::= &A | B | C | D | E | F | G | H | I | J 
                       | K | L | M | N | O \\
              &     &  | P | Q | R | S | T | U | V | W | X
                       | Y | Z | \_ \\
<数字>      & ::= &0 | 1 | 2 | 3 | 4 | 5 | 6 | 7 | 8 | 9 \\
<命令1>    & ::= &NO | | EI | DI | RET |RETI | HALT \\
<命令2>    & ::= &SHLA | SHLL | SHRA | SHRL | PUSH | POP \\
<命令3>    & ::= &IN | OUT \\
<命令4>    & ::= &LD | ADD | SUB | CMP | AND | OR |XOR \\
<命令5>    & ::= &ST \\
<命令6>    & ::= &JMP | JZ | JC | JM | JNZ | JNC | JNM | CALL \\
<命令7>    & ::= &EQU | DS | ORG \\
<命令8>    & ::= &DC \\
<インデクス>& ::= &G1 | G2 \\
<レジスタ>  & ::= &G0 | G1 | G2 | SP \\
<改行>      & ::= &'$\backslash$n' \\
<空白>      & ::= &'~' |'$\backslash$t' \\
\end{tabular}}
%\caption{アセンブラの文法}
%\label{fig:appB:syntax}
\end{center}
%\end{figure*}

\newpage
\section{プログラム例}
%\figref{appB:exp1}にローマ字で名前をSIOに出力するプログラムの例を示します.
以下にローマ字で名前(sigemura)をSIOに出力するプログラムの例を示します.

%\begin{figure*}[btp]
\begin{center}
\begin{verbatim}
ADR  CODE          Label   Instruction             Comment              Page(1)

02                 SIOD    EQU     02H            
03                 SIOS    EQU     03H            
00                 
00  17 00          START   LD      G1,#0          
02  19 15          L0      LD      G2,DATA,G1     
04  5B 00                  CMP     G2,#0          
06  A4 14                  JZ      END            
08  C0 03          L1      IN      G0,SIOS        
0A  63 80                  AND     G0,#80H        
0C  A4 08                  JZ      L1             
0E  CB 02                  OUT     G2,SIOD        
10  37 01                  ADD     G1,#1          
12  A0 02                  JMP     L0             
14  FF             END     HALT                   
15                 
15  73 69 67 65    DATA    DC      "sigemura"     
19  6D 75 72 61 
1D  0D 0A                  DC      0DH,0AH        
1F  00                     DC      0              ; 終わりの印
\end{verbatim}
%\caption{ローマ字で名前をSIOに出力するプログラム}
%\label{fig:appB:exp1}
\end{center}
%\end{figure*}

\section{アセンブラの実行例}
TeC7を用い
ローマ字で名前をSIOに出力するプログラムをアセンブル・実行した例を示します.

%\begin{figure*}[btp]
\begin{center}
\begin{verbatim}
$ tasm7 sigemura.t7                            <--- アセンブル
アセンブル成功
結果は [sigemura.lst] と [sigemura.bin] に格納しました。
...省略...

$ twrite7 sigemura.bin                         <--- ダウンロード
[00][20][17][00][19][15][5b][00][a4][14][c0][03][63][80][a4][08][cb][02][37]
[01][a0][02][ff][73][69][67][65][6d][75][72][61][0d][0a][00]
$ screen /dev/cu.usbserial-xxxxxxxxx           <--- 通信ソフト起動
sigemura                                       <--- 実行結果
^A^\                                           <--- 通信ソフトの終了
[screen is terminating]
\end{verbatim}
\end{center}
%\end{figure*}

%\newpage
%\section{IPLプログラム}\label{iplsrc}
%TeCのROM に格納されているIPLプログラムのアセンブルリストを
%参考のため以下に示します.
%TeCのメモリ空間はE0H〜FFH番地がROMになっており,
%以下のIPLプログラムが予め格納されています.
%IPLプログラムは,パソコン上でtasm7を使用してクロス開発さ
%れた機械語プログラムをシリアル入出力から受信してメモリに格納します.
%
%%\begin{figure*}[btp]
%\begin{center}
%\begin{verbatim}
%ADR  CODE          Label   Instruction             Comment              Page(1)
%
%02                 SIOD    EQU     02H            
%03                 SIOS    EQU     03H            
%E0                         ORG     0E0H           
%E0  1F DC          IPL     LD      SP,#0DCH       ; 割込みベクタの直前がスタック
%E2  B0 F6                  CALL    READ           ; ロードアドレスを入力
%E4  D0                     PUSH    G0             
%E5  D6                     POP     G1             
%E6  B0 F6                  CALL    READ           ; 長さを入力
%E8  D0                     PUSH    G0             
%E9  DA                     POP     G2             
%EA  A4 FF          LOOP    JZ      STOP           
%EC  B0 F6                  CALL    READ           
%EE  21 00                  ST      G0,0,G1        
%F0  37 01                  ADD     G1,#1          
%F2  4B 01                  SUB     G2,#1          
%F4  A0 EA                  JMP     LOOP           
%F6                 ;
%F6  C0 03          READ    IN      G0,SIOS        
%F8  63 40                  AND     G0,#40H        
%FA  A4 F6                  JZ      READ           
%FC  C0 02                  IN      G0,SIOD        
%FE  EC                     RET                    
%FF                 ;
%FF  FF             STOP    HALT                   ; PC がゼロになって止まる
%\end{verbatim}
%\end{center}
%%\end{figure*}

\newpage
\section{機械語プログラムファイル形式}
\label{appB:fileformat}
tasm7 が出力する機械語プログラムの形式は次の通りです.

\begin{quote}
{\small\bf\begin{tabular}{|c|}
\multicolumn{1}{c}{}\\
\hline
ロードアドレス(1バイト) \\
\hline
プログラム長(1バイト) \\
\hline
\\
...\\
機械語(1バイト以上) \\
...\\
\\
\hline
\end{tabular}}
\end{quote}

ファイルの1バイト目は,
機械語プログラムのTeC主記憶上の配置アドレスです.
機械語は,このバイトで表される番地を先頭にして主記憶に書き込まれます.
ロードアドレスは,アセンブラのORG命令で指示することができます.
ORG命令を使用しない場合のロードアドレスは,0番地になります.
ファイルの2バイト目は,機械語プログラムの長さです.
3バイト目以降の機械語の長さをバイト単位で表します.

