\chapter{はじめに}

\section{この科目で学ぶこと}
現代,コンピュータと呼ばれているものはスーパーコンピュータと呼ば
れる高価で高性能なものから,マイコンと呼ばれ炊飯器やエアコンに組
み込まれている小型のものまで,全て同じ原理に基づき動作しています.

この原理は,1946年に米国の数学者フォン・ノイマン(Von Neumann)
が提案したと言われています.現代のコンピュータの,ほぼ,全てのもの
は,ノイマンの提案した同じ原理を使用しており「ノイマン型コンピュー
タ」と呼ばれます.

「ノイマン型コンピュータ」が出現して既に約80年の時間が経過しま
したが,未だにこれを上回る「自動機械」の構成方法は発明されていま
せん.だから,パソコンのような一般の人がコンピュータだと考えてい
る装置だけでなく,炊飯器やエアコンの制御装置のようなコンピュータ
らしくない装置まで,「自動機械」が必要なところには全て「ノイマン型
コンピュータ」が使用されているのです.
この先も,しばらく「ノイマン型コンピュータ」の時代が続くでしょう.

この教科書は,教育用コンピュータ(TeC)を用いて,
\emph{ノイマン型コンピュータの動作原理}をしっかり勉強できるようになっています.
ここでしっかり勉強しておけば,将来,どんな新型の
コンピュータに出会ったとしても,恐れることはありません.
所詮,「ノイマン型コンピュータ」の一種ですから,皆さんはその正体
を簡単に見抜くことができるはずです.

「ノイマン型コンピュータ」の原理をきちんと理解しておくことは,皆
さんにとって,寿命の長いエンジニアになるための大切なステップとなり
ます.しっかり,がんばって下さい.

\section{勉強の進め方}

この教科書は,高専や工業高校の1年生と2年生が,
教育用コンピュータTeCを教材に,
ノイマン型コンピュータの基本を学ぶことを想定して作ってあります.
1章から5章までを1年で学び,
残りを2年生(半年〜1年間)で学びます.

1年生では,(1)コンピュータの内部で情報を表現する方法,
(2)TeCの組み立て,(3)TeCの基本的なプログラミングを学びます.
2年生では,TeCを用いた高度なプログラミングを学びます.

\section{教材用コンピュータ}

私達の身近にあるパーソナルコンピュータ(パソコン)や
スマートフォン(スマホ)のようなコンピュータシステムは,
高度で複雑すぎて「ノイマン型コンピュータ」の
原理を学ぶための教材としては適していません.

原理を学ぶのに適した単純で小さなコンピュータ(マイコン)を,
専用に開発しました.このマイコンはTeC(Tokuyama Educational
Computer : 徳山高専教育用コンピュータ,\figref{chap1:TeC})と呼ばれます.

\myfigureNA{btp}{scale=0.35}{Img/TeC7c.jpg}
{教育用コンピュータTeC7c}{chap1:TeC}

TeCの特徴は単純で小さなことです.
単純で小さなことで,次のようなメリットがあります.

\begin{itemize}
\item[単純]
ノイマン型コンピュータの本質的な部分だけを勉強しやすい.
現代のパソコン等は,勉強するには難しすぎる.
\item[小型]
自宅に持ち帰り宿題ができる.
授業時間外でも,納得がいくまで色々と試してみることができる.
\end{itemize}

\section{TeCの種類}

TeCは2003年から現在まで20年以上使用されてきました.
その間に色々な改良がされました.
現在では大きく四種類のTeCがあります.

\begin{itemize}
\item TeC6 \\
2003年から使用を開始しました.
TeCの基本的な機能は現在と変わりません.
\item TeC7a \\
2011年から使用を開始しました.
TeCで使用できる機能が少し増えました.
大きな変更は,
ジャンパーを切り替えることでTaCという名前の別のコンピュータに
切換えることができるようになったことです.
TaCは高学年や専攻科で使用します.
TaCモードではPC用のキーボードやディスプレイを接続してPCのように使用できます.
\item TeC7c \\
2018年から使用を開始したTeC7aの改良版です.
Bluetoothを用いて接続したスマホなどを
キーボードやディスプレイの代替として使用できるようになりました.
\item TeC7d \\
2019年から使用を開始したTeC7cの改良版です.
ケースに入れたまま使用できるように電源コネクタを垂直型に変更した他に,
電源を入れた状態でもマイクロSDカードを交換できるような改良がされました.
\end{itemize}

本書はTeC7cとTeC7dを基本に解説します.
TeC6について知りたい場合は本書の古いバージョン「TeC教科書 Ver. 3.2.2」
(\url{https://github.com/tctsigemura/TecTextBook/blob/v3.2.2/tec.pdf})
を,
TeC7aについて知りたい場合は「TeC教科書 Ver. 4.0.5」
(\url{https://github.com/tctsigemura/TecTextBook/blob/v4.0.5/tec.pdf})
を見て下さい.
