\renewcommand{\myincludegraphics}[2]{\includegraphics[#2]{appA/#1}}

\chapter{電子オルゴールプログラムの実行例}
以下は,\figref{hexlist}の電子オルゴールプログラムを入力して実行する手順です.
少し,音痴なオルゴールですがTeCから音楽が流れます.

\section{プログラムの打ち込み}
電子オルゴールプログラムは予め作ってあります.
(\figref{hexlist}の16進数のリストです.)
これを打ち込みます.
リストは,$00_{16}$番地に$17_{16}$,
$01_{16}$番地に$2E_{16}$,...,$0F_{16}$番地に$01_{16}$,
$10_{16}$番地に$A4_{16}$,...を
打ち込むことを表わしています.

\myfigureNA{btp}{width=\textwidth}{doremi.pdf}
{電子オルゴールプログラムの16進数リスト}{hexlist}

打ち込み手順は次の通りです.
操作方法は,\ref{operation}を参照してください.
\begin{enumerate}
\item ロータリースイッチをMMの位置に合わせます.
\item アドレスランプに$00_{16}$(2進数で$0000~0000_{2}$)をセットします.
\item 16進数を順に打ち込みます.\\
 最初は,$17_{16}$ですから,データスイッチに$0001~0111_{2}$をセットして
WRITEスイッチを押します.
次は,$2E_{16}$ですから,今度はデータスイッチに$0010~1110_{2}$をセットして
WRITEスイッチを押します.
以降,同様にリストの最後まで打ち込みます.
\item 最後まで打ち込んだら,最初からもう一度データを確認します.\\
 アドレスランプに$00_{16}$をセットしデータを確認します.
合っていたら,INCAスイッチを押して次の番地に進みデータを確認します.
間違っていた場合は,正しい値をデータスイッチにセットして
WRITEスイッチを押します.
\end{enumerate}

\section{プログラムの実行}
このプログラムの実行開始番地は$00_{16}$です.
PCに$00_{16}$をセットし,プログラムの実行を開始します.
手順は次の通りです.
\begin{enumerate}
\item ロータリースイッチをPCの位置に合わせます.
\item データスイッチに$0000~0000_{2}$をセットしWRITEスイッチを押します.
\item BREAKスイッチ,STEPスイッチが下に倒れていることを確認します.\\
(下に倒れていなかったら,下に倒します.)
\item RUNボタンを押し,プログラムの実行を開始します.
\end{enumerate}

\section{残念ですが...}
苦労して打ち込んだプログラムは保存しておきたいのですが,
TeCにはその機能がありません.
電源を切るとプログラムが消えてしまいます.
残念ですが,毎回,打ち込む必要があります.

\section{曲データの変更}
打ち込んだ16進数のリストの構成は,
$00_{16}$番地から$2F_{16}$番地までにプログラム,
$30_{16}$番地から$65_{16}$番地までに「ドレミの歌」(途中まで)の音楽データ,
$66_{16}$番地に音楽データの終了を表す$00_{16}$となっています(\figref{hexlist}参照).
音楽データの部分を他の曲のデータに置き換えると別の曲を鳴らすことができます.

音楽データは2バイトで一つの音を表すようになっています.
四分音符の各高さの音のデータは下表の通りです.
1バイト目が音の長さを調整するパラメータ,
2バイト目が音の高さを制御するパラメータです.
1バイト目の値は音の高さ毎に意味が変化します.

例えば,$30_{16}$番地と$31_{16}$番地の$C5_{16}$と$E8_{16}$は,
付点四分音符のドの音を表わしています.
四分音符のドのデータは$83_{16}$と$E8_{16}$ですので,
付点四分音符になるように1バイト目の値を1.5倍にして,
このデータができました.

$FF_{16}$より長い値は指定できませんので,
指定できない長い音は同じデータの繰り返しで表現して下さい.
(休符は表現できません.^^;;;)
少々(かなり?),制限が多いですがプログラムの打ち込みの練習になりますので,
是非,別の曲のデータを作って打ち込んでみて下さい.

{\small
\begin{center}
\fbox{\parbox{7.5cm}{
{\bf 電子オルゴールデータの意味} \\
\begin{center}
\begin{tabular}{l l l}
音      & 長さ & 高さ \\
ド     & 83 & E8 \\
ド\#   & 8B & DD \\
レ     & 93 & D1 \\
レ\#   & 9C & C5 \\
ミ     & A5 & BA \\
ファ   & AF & AF \\
ファ\# & B9 & A6 \\
ソ     & C4 & 9C \\
ソ\#   & D0 & 93 \\
ラ     & DC & 8B \\
ラ\#   & E9 & 83 \\
シ     & F7 & 7C \\
ド     & FF & 75 \\
\end{tabular}
\end{center}
}}
\end{center}
}
