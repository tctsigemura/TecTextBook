%\documentclass[dvipdfmx]{beamer}      % platex の場合
\documentclass[handout]{beamer}        % lualatex の場合
\usepackage{mySld}
\usepackage{multirow}

\begin{document}
\title{基礎コンピュータ工学\\第2章 情報の表現\\
      (パート6:論理演算と基本回路)}
\date{}

\begin{frame}
  \titlepage
  \centerline{\url{https://github.com/tctsigemura/TecTextBook}}
  \vfill
  \centerline{本スライドの入手:
    \raisebox{-7mm}{\includegraphics[scale=0.3]{../Img/QRs2_5.png}}}
\end{frame}

%==============================================================================
%\begin{frame}
%  \frametitle{}
%\end{frame}

\section{情報の表現}
%==============================================================================
\begin{frame}
  \frametitle{コンピュータの基本回路}
  \emph{論理回路}を組合せてコンピュータは製作される.
  \begin{itemize}
  \item 電気のONとOFFだけ用いて情報を表現する.
  \item \emph{ON/OFF}をビット\emph{1/0}に対応付ける.
  \item これは\emph{論理値}(True/False,真/偽,Yes/No)と同じ.
  \item 論理値を対象とする演算を\emph{論理演算}と言う.
  \item 論理演算を計算する回路を\emph{論理回路}と言う。
  \end{itemize}
  \vfill
  \centerline{\includegraphics[scale=0.7]{../Tikz/lic.pdf}}
\end{frame}

%==============================================================================
\begin{frame}
  \frametitle{論理ICで作った手作りコンピュータ}
  \emph{徳山高専卒業研究}で約40年前に製作された手作りコンピュータ
  \vfill
  \begin{tabular}{c c}
  \multirow{2}{*}{\includegraphics[scale=0.04]{../Img/uCom16_1.jpg}}
  &\includegraphics[scale=0.03]{../Img/uCom16_2.jpg} \\
  &\includegraphics[scale=0.03]{../Img/uCom16_3.jpg}
  \end{tabular}
\end{frame}

%==============================================================================
\begin{frame}
  \frametitle{論理ICで回路を作る}
  \centerline{\includegraphics[scale=0.6]{../Keynote/majority-crop.pdf}}
\end{frame}

%==============================================================================
\begin{frame}
  \frametitle{現代の手作りコンピュータ}
  \emph{FPGA}の中に論理IC数万個に相当する回路が書き込める.
  \vfill
  \centerline{\includegraphics[scale=0.26]{../Img/TeC7c.jpg}}
\end{frame}

%==============================================================================
\begin{frame}
  \frametitle{基本的な論理回路(1)}
  \emph{論理積(AND)} --- 「かつ」\\
  \vfill
  2ビット入力し,両方が1のときだけ1を出力する.
  \centerline{\includegraphics[scale=1.4]{../Tikz/and.pdf}}
  \begin{quote}
    A \emph{かつ} B が1 なら 1\\
    A \emph{AND} B が1 なら 1
  \end{quote}
  \vfill
\end{frame}

%==============================================================================
\begin{frame}
  \frametitle{基本的な論理回路(2)}
  \emph{論理和(OR)} --- 「または」\\
  \vfill
  2ビット入力し,どちらかが1のとき1を出力する.
  \centerline{\includegraphics[scale=1.4]{../Tikz/or.pdf}}
  \begin{quote}
    A \emph{または} B が1 なら 1\\
    A \emph{OR} B が1 なら 1
  \end{quote}
  \vfill
\end{frame}

%==============================================================================
\begin{frame}
  \frametitle{基本的な論理回路(3)}
  \emph{否定(NOT)} --- 「ではない」\\
  \vfill
  1ビット入力し,入力とは逆の論理値をを出力する.
  \centerline{\includegraphics[scale=1.4]{../Tikz/not.pdf}}
  \begin{quote}
    A が 1 \emph{ではない} なら 1\\
  \end{quote}
  \vfill
\end{frame}

%==============================================================================
\begin{frame}
  \frametitle{基本的な論理回路(4)}
  \emph{排他的論理和(XOR)} --- 「異なる」\\
  \vfill
  2ビット入力し,二つが異なるなら1を出力する.
  \centerline{\includegraphics[scale=1.4]{../Tikz/xor.pdf}}
  \begin{quote}
    A と B が異なるなら 1\\
  \end{quote}
  \vfill
\end{frame}

%==============================================================================
\begin{frame}
  \frametitle{基本的な論理回路(5)}
  \emph{NOTとANDの組合せ(NAND)} \\
  \vfill
  \centerline{\includegraphics[scale=1.3]{../Tikz/nand.pdf}}
  \vfill
\end{frame}

%==============================================================================
\begin{frame}
  \frametitle{基本的な論理回路(6)}
  \emph{NOTとORの組合せ(NOR)} \\
  \vfill
  \centerline{\includegraphics[scale=1.3]{../Tikz/nor.pdf}}
  \vfill
\end{frame}

%==============================================================================
\begin{frame}
  \frametitle{演算回路(1)}
  \emph{半加算器} \\
  1桁の2進数を二つ入力し,0,1,2のどれかを出力する.
  \vfill
  \centerline{\includegraphics[scale=1.3]{../Tikz/ha.pdf}}
  \vfill
\end{frame}

%==============================================================================
\begin{frame}
  \frametitle{演算回路(2)}
  \emph{全加算器} \\
  1桁の2進数を三つ入力し,0,1,2,3のどれかを出力する.
  \vfill
  \centerline{\includegraphics[scale=1.3]{../Tikz/fa.pdf}}
  \vfill
\end{frame}

%==============================================================================
\begin{frame}
  \frametitle{演算回路(3)}
  \emph{4ビット加算器} \\
  4桁の2進数を二つ入力し,和を計算する
  \vfill
  \centerline{\includegraphics[scale=0.8]{../Tikz/adder.pdf}}
  \vfill
\end{frame}

%==============================================================================
\begin{frame}
  \frametitle{演算回路(4)}
  \emph{4ビット 1の補数器} \\
  4桁の2進数を入力し,1の補数を計算する
  \vfill
  \centerline{\includegraphics[scale=1.0]{../Tikz/onesc.pdf}}
  \vfill
\end{frame}

%==============================================================================
\begin{frame}
  \frametitle{演算回路(5)}
  \emph{4ビット 2の補数器} \\
  4桁の2進数を入力し,2の補数を計算する
  \vfill
  \centerline{\includegraphics[scale=0.7]{../Tikz/twosc.pdf}}
  \vfill
\end{frame}

%==============================================================================
\begin{frame}
  \frametitle{記憶回路}
  \emph{RSフリップフロップ} \\
  直前の状態を記憶する回路
  \vfill
  \centerline{\includegraphics[scale=1.3]{../Tikz/rsff.pdf}}
  \vfill
\end{frame}

\end{document}
