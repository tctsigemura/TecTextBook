%\documentclass[dvipdfmx]{beamer}      % platex の場合
\documentclass[handout]{beamer}        % lualatex の場合
\usepackage{mySld}
\usepackage{multicol}

\begin{document}
\title{基礎コンピュータ工学\\第5章 機械語プログラミング\\
       (パート6:残りの条件分岐命令と条件判断の演習)}
\date{}

\begin{frame}
  \titlepage
  \centerline{\url{https://github.com/tctsigemura/TecTextBook}}
  \vfill
  \centerline{本スライドの入手:
    \raisebox{-7mm}{\includegraphics[scale=0.3]{../Img/QRs5_6.png}}}
\end{frame}

%==============================================================================
%\begin{frame}
%  \frametitle
%  \tableofcontents
%\end{frame}

\section{条件ジャンプ命令}
%==============================================================================
\begin{frame}
  \frametitle{ジャンプ命令(7種類)の残り3種類}
  \begin{description}
  \item[無条件ジャンプ命令:]\emph{プログラムの流れ}を指定のアドレスに飛ばす.
    \begin{itemize}
      \item \emph{JMP(Jump)命令}:いつもジャンプする.
    \end{itemize}
    \vfill
  \item[条件ジャンプ命令:]ある条件のときだけジャンプする.
    \begin{itemize}
      \item \emph{JZ(Jump on Zero)命令}:$Z=1$ ならジャンプ
        \vfill
      \item \emph{JC(Jump on Carry)命令}:$C=1$ ならジャンプ
        \vfill
      \item \emph{JM(Jump on Minus)命令}:$S=1$ ならジャンプ
        \vfill
      \item \emph{JNZ(Jump on Not Zero)命令}:$Z=0$ ならジャンプ
        \vfill
      \item \emph{JNC(Jump on Not Carry)命令}:$C=0$ ならジャンプ
        \vfill
      \item \emph{JNM(Jump on Not Minus)命令}:$S=0$ ならジャンプ
    \end{itemize}
  \end{description}
  \vfill
\end{frame}

%==============================================================================
\begin{frame}
  \frametitle{JNZ(Jump on Not Zero)命令}
  Zフラグが0なら(計算結果が0でないなら)ジャンプする.
  \vfill
  \begin{description}
  \item[フラグ:] 変化しない.
    \vfill

  \item[ニーモニック:]\texttt{JNZ EA}~~~~~~~~~\texttt{(if(Z=0) PC ← EA)}
    \vfill

  \item[命令フォーマット:] 2バイトの長さを持つ.\\
    \twoByte{$1011_2$}{$01_2$~\XR}{\A}
    \vfill

  \item[フローチャート:] ある程度,自由にアレンジしてよい.\\
    \centerline{\includegraphics[scale=0.7]{../Tikz/jnz.pdf}}
  \end{description}
  \vfill
\end{frame}

%==============================================================================
\begin{frame}
  \frametitle{JNZ命令の使用例(JZ命令との比較)}
  ループを3回,繰り返すプログラム\\
  \vfill
  \begin{minipage}{0.4\columnwidth}
    \centerline{\includegraphics[scale=0.6]{../Tikz/flow0C.pdf}}
    \centerline{\includegraphics[scale=0.6]{../Tikz/flow0B.pdf}}
  \end{minipage}
  \begin{minipage}{0.59\columnwidth}
    {\ttfamily\footnotesize\begin{center}
      \begin{tabular}{|l|l|l|l l|} \hline
        番地 & 機械語 & ラベル & \multicolumn{2}{|c|}{ニーモニック} \\
        \hline
        00 & 10 07 &           & LD   & G0,THREE              \\
        02 & 40 08 &  LOOP     & SUB  & G0,ONE                \\
        04 & B4 02 &           & JNZ  & LOOP                  \\
        06 & FF    &           & HALT &                       \\
        07 & 03    &  THREE    & DC   & 3                     \\
        08 & 01    &  ONE      & DC   & 1                     \\
        \hline
      \end{tabular}
    \end{center}}
    \vspace{0.5cm}
    {\ttfamily\footnotesize\begin{center}
      \begin{tabular}{|l|l|l|l l|} \hline
        番地 & 機械語 & ラベル & \multicolumn{2}{|c|}{ニーモニック} \\
        \hline
        00 & 10 09 &           & LD   & G0,THREE              \\
        02 & 40 0A &  LOOP     & SUB  & G0,ONE                \\
        04 & A4 08 &           & JZ   & STOP                  \\
        06 & A0 02 &           & JMP  & LOOP                  \\
        08 & FF    &  STOP     & HALT &                       \\
        09 & 03    &  THREE    & DC   & 3                     \\
        0A & 01    &  ONE      & DC   & 1                     \\
        \hline
      \end{tabular}
    \end{center}}
  \end{minipage}
  \vfill
\end{frame}

%==============================================================================
\begin{frame}
  \frametitle{JNC(Jump on Not Carry)命令}
  Cフラグが0なら(オーバーフローしていないなら)ジャンプする.
  \vfill
  \begin{description}
  \item[フラグ:] 変化しない.
    \vfill

  \item[ニーモニック:]\texttt{JNC EA}~~~~~~~~~\texttt{(if(C=0) PC ← EA)}
    \vfill

  \item[命令フォーマット:] 2バイトの長さを持つ.\\
    \twoByte{$1011_2$}{$10_2$~\XR}{\A}
    \vfill

  \item[フローチャート:] ある程度,自由にアレンジしてよい.\\
    \centerline{\includegraphics[scale=0.7]{../Tikz/jnc.pdf}}
  \end{description}
  \vfill
\end{frame}

%==============================================================================
\begin{frame}
  \frametitle{JNM(Jump on Not Minus)命令}
  Sフラグが0なら(正かゼロなら)ジャンプする.
  \vfill
  \begin{description}
  \item[フラグ:] 変化しない.
    \vfill

  \item[ニーモニック:]\texttt{JNM EA}~~~~~~~~~\texttt{(if(S=0) PC ← EA)}
    \vfill

  \item[命令フォーマット:] 2バイトの長さを持つ.\\
    \twoByte{$1011_2$}{$11_2$~\XR}{\A}
    \vfill

  \item[フローチャート:] ある程度,自由にアレンジしてよい.\\
    \centerline{\includegraphics[scale=0.7]{../Tikz/jnm.pdf}}
  \end{description}
  \vfill
\end{frame}

%==============================================================================
%\begin{frame}
%  \frametitle{条件判断1(JNZ使用)}
%  計算結果により処理をするかしないか変化する例
%  \vfill
%  \begin{minipage}{0.49\columnwidth}
%    \begin{itembox}[l]{\footnotesize 計算結果がゼロなら「処理」\emph{しない}}
%      \centerline{\includegraphics[scale=0.6]{../Tikz/flow2AB.pdf}}
%    \end{itembox}
%  \end{minipage}
%  \begin{minipage}{0.5\columnwidth}
%    \begin{itembox}[l]{\footnotesize 計算結果がゼロなら「処理」\emph{する}}
%      \centerline{\includegraphics[scale=0.6]{../Tikz/flow2AC.pdf}}
%    \end{itembox}
%  \end{minipage}
%  \vfill
%\end{frame}

%==============================================================================
\begin{frame}
  \frametitle{条件判断の例(JNM使用)}
  絶対値を求めるプログラム(例題5-1の改良)\\
  \vfill
  \begin{minipage}{0.49\columnwidth}
    \centerline{\includegraphics[scale=0.5]{../Tikz/flow1.pdf}}
  \end{minipage}
  \begin{minipage}{0.5\columnwidth}
    {\ttfamily\scriptsize\begin{center}
      \begin{tabular}{|l|l|l|l l|}
        \hline
        番地 & 機械語 & ラベル & \multicolumn{2}{|c|}{ニーモニック} \\
        \hline
        00 & 10 0E & START& LD   & G0,N    \\
        02 & 40 0D &      & SUB  & G0,ZERO \\
        04 & BC 0A &      & JNM  & L1      \\
        06 & 10 0D &      & LD   & G0,ZERO \\
        08 & 40 0E &      & SUB  & G0,N    \\
        0A & 20 0F & L1   & ST   & G0,M    \\
        0C & FF    &      & HALT &         \\
        0D & 00    & ZERO & DC   & 0       \\
        0E & FF    & N    & DC   & -1      \\
        0F & 00    & M    & DS   & 1       \\
        \hline
      \end{tabular}
    \end{center}}
  \end{minipage}
  \vfill
  \begin{itemize}
  \item JNMを使用したほうが短くなる.
  \end{itemize}
\end{frame}

%==============================================================================
\begin{frame}
  \frametitle{まとめ}
  \emph{●学んだこと}\\
    必須ではないがあると便利なJNZ,JNC,JNM命令
  \vfill
  \emph{●条件判断の演習}
    \begin{enumerate}
    \item[1.] プログラムの作成手順を再度確認
      \begin{enumerate}
      \item[(1)] フローチャートを描く.
      \item[(2)] フローチャートを基にニーモニックを書く.
      \item[(3)] アドレスを決める.
      \item[(4)] 機械語を作る.
      \end{enumerate}
      \vfill
  \item[2.] 演習(宿題)
      \begin{enumerate}
      \item[(1)] N番地の値がゼロならM番地にゼロを,
        そうでなければM番地に1を格納するプログラム
        \begin{itemize}
        \item LD命令はフラグを変化させないので...
        \item 前回の「条件判断2」のパターンを利用
        \end{itemize}
        \vfill
      \item[(2)] N番地の値とM番地の値で,
        大きい方をL番地に格納するプログラム
        \begin{itemize}
        \item 値は符号付きの数値とする.
        \item 比較は引き算でできる.
        \end{itemize}
      \end{enumerate}
  \end{enumerate}
  \vfill
\end{frame}

\end{document}
