%\documentclass[dvipdfmx]{beamer}      % platex の場合
\documentclass[handout]{beamer}        % lualatex の場合
\usepackage{mySld}

\begin{document}
\title{基礎コンピュータ工学\\第2章 情報の表現\\
       (パート3:2進数の計算と2の補数)}
\date{}

\begin{frame}
  \titlepage
  \centerline{\url{https://github.com/tctsigemura/TecTextBook}}
  \vfill
  \centerline{本スライドの入手:
    \raisebox{-7mm}{\includegraphics[scale=0.3]{../Img/QRs2_3.png}}}
\end{frame}

%==============================================================================
%\begin{frame}
%  \frametitle{}
%\end{frame}

\section{情報の表現}
%==============================================================================
\begin{frame}
  \frametitle{2進数の和差の計算}

\emph{10進数の場合}を思い出してみる.

  \begin{itemize}
  \item 9より大きくなる時に\emph{桁上げ}が発生する.

    \texttt{
    \begin{tabular}{l l}
        & 103 \\
      + & 104 \\
      \cline{1-2}
        & 207
    \end{tabular}~~
    \begin{tabular}{l l}
        & 105 \\
      + & 107 \\
      \cline{1-2}
        & 212
    \end{tabular}~~
    \begin{tabular}{l l}
        & 135 \\
      + & 127 \\
      \cline{1-2}
        & 262
    \end{tabular}~~
    \begin{tabular}{l l}
        & 155 \\
      + & 167 \\
      \cline{1-2}
        & 322
    \end{tabular}~~
    \begin{tabular}{l l}
        & 099 \\
      + & 001 \\
      \cline{1-2}
        & 100
    \end{tabular}
    }

    \vfill
  \item \emph{桁借り}では10借りてくる.

    \texttt{
    \begin{tabular}{l l}
        & 207 \\
      - & 104 \\
      \cline{1-2}
        & 103
    \end{tabular}~~
    \begin{tabular}{l l}
        & 212 \\
      - & 107 \\
      \cline{1-2}
        & 105
    \end{tabular}~~
    \begin{tabular}{l l}
        & 262 \\
      - & 127 \\
      \cline{1-2}
        & 135
    \end{tabular}~~
    \begin{tabular}{l l}
        & 322 \\
      - & 167 \\
      \cline{1-2}
        & 155
    \end{tabular}~~
    \begin{tabular}{l l}
        & 100 \\
      - & 001 \\
      \cline{1-2}
        & 099
    \end{tabular}
    }
  \end{itemize}
\end{frame}

%==============================================================================
\begin{frame}
  \frametitle{2進数の和差の計算}

\emph{2進数の場合}は以下のようになる.
  \begin{itemize}
  \item 1より大きくなる時に\emph{桁上げ}が発生する.

    \texttt{
    \begin{tabular}{l l}
        & 010 \\
      + & 001 \\
      \cline{1-2}
        & 011
    \end{tabular}~~
    \begin{tabular}{l l}
        & 001 \\
      + & 001 \\
      \cline{1-2}
        & 010
    \end{tabular}~~
    \begin{tabular}{l l}
        & 010 \\
      + & 011 \\
      \cline{1-2}
        & 101
    \end{tabular}~~
    \begin{tabular}{l l}
        & 011 \\
      + & 001 \\
      \cline{1-2}
        & 100
    \end{tabular}~~
    \begin{tabular}{l l}
        & 011 \\
      + & 011 \\
      \cline{1-2}
        & 110
    \end{tabular}
    }

    \vfill
  \item \emph{桁借り}では2借りてくる.
    \texttt{
    \begin{tabular}{l l}
        & 011 \\
      - & 001 \\
      \cline{1-2}
        & 010
    \end{tabular}~~
    \begin{tabular}{l l}
        & 010 \\
      - & 001 \\
      \cline{1-2}
        & 001
    \end{tabular}~~
    \begin{tabular}{l l}
        & 101 \\
      - & 011 \\
      \cline{1-2}
        & 010
    \end{tabular}~~
    \begin{tabular}{l l}
        & 100 \\
      - & 001 \\
      \cline{1-2}
        & 011
    \end{tabular}~~
    \begin{tabular}{l l}
        & 110 \\
      - & 011 \\
      \cline{1-2}
        & 011
    \end{tabular}
    }

  \end{itemize}
\end{frame}

%==============================================================================
\begin{frame}[fragile]
  \frametitle{2進数の和差の計算(問題)}
  \emph{問題8:}10進数の計算と2進数の計算をしなさい.
    \begin{minipage}{0.48\columnwidth}
      \begin{itembox}[l]{3+8}
        \texttt{
          \begin{tabular}{r r}
            \multicolumn{2}{c}{10進} \\
              &  3 \\
            + &  8 \\
            \cline{1-2}
              &
          \end{tabular}~~~
          \begin{tabular}{r r}
            \multicolumn{2}{c}{2進} \\
              & 0011 \\
            + & 1000 \\
            \cline{1-2}
              &
          \end{tabular}
        }
      \end{itembox}
    \end{minipage}
    \begin{minipage}{0.48\columnwidth}
      \begin{itembox}[l]{5+7}
        \texttt{
          \begin{tabular}{r r}
            \multicolumn{2}{c}{10進} \\
              &  5 \\
            + &  7 \\
            \cline{1-2}
              &
          \end{tabular}~~~
          \begin{tabular}{r r}
            \multicolumn{2}{c}{2進} \\
              & 0101 \\
            + & 0111 \\
            \cline{1-2}
              &
          \end{tabular}
        }
      \end{itembox}
    \end{minipage}\\
    \begin{minipage}{0.48\columnwidth}
      \begin{itembox}[l]{11-8}
        \texttt{
          \begin{tabular}{r r}
            \multicolumn{2}{c}{10進} \\
              & 11 \\
            - &  8 \\
            \cline{1-2}
              &
          \end{tabular}~~~
          \begin{tabular}{r r}
            \multicolumn{2}{c}{2進} \\
              & 1011 \\
            - & 1000 \\
            \cline{1-2}
              &
          \end{tabular}
        }
      \end{itembox}
    \end{minipage}
    \begin{minipage}{0.48\columnwidth}
      \begin{itembox}[l]{12-7}
        \texttt{
          \begin{tabular}{r r}
            \multicolumn{2}{c}{10進} \\
              & 12 \\
            - &  7 \\
            \cline{1-2}
              &
          \end{tabular}~~~
          \begin{tabular}{r r}
            \multicolumn{2}{c}{2進} \\
              & 1100 \\
            - & 0111 \\
            \cline{1-2}
              &
          \end{tabular}
        }
      \end{itembox}
    \end{minipage}
\end{frame}

%==============================================================================
\begin{frame}
  \frametitle{負数の表現}
負の数を2進数でどのようにビットで表現するか約束する.

\begin{enumerate}
\item[(1)] \emph{符号付き絶対値表現}\\
  左端のビットを符号($+$/$-$)として使用する.
  \begin{itembox}[l]{4ビット符号付き絶対値表現の例}
    \begin{minipage}{0.5\columnwidth}
      \begin{tabular}{ r | r || r | r }
        \hline
        \hline
        負数 & 2進数    & 正数 & 2進数 \\
        \hline
        $-7$ & $1111_2$ & $+7$ & $0111_2$ \\
        $-6$ & $1110_2$ & $+6$ & $0110_2$ \\
        $-5$ & $1101_2$ & $+5$ & $0101_2$ \\
        ...  & ...      & ...  & ... \\
        $-1$ & $1001_2$ & $+1$ & $0001_2$ \\
        $-0$ & $1000_2$ & $+0$ & $0000_2$ \\
      \end{tabular}
    \end{minipage}
    \begin{minipage}{0.5\columnwidth}
      \begin{itemize}
      \item 4ビットで$-7$から$+7$の範囲を表現できる.
      \item $0$の表現が二つある($-0$と$+0$).
      \end{itemize}
    \end{minipage}
  \end{itembox}
\end{enumerate}
\end{frame}

%==============================================================================
\begin{frame}
  \frametitle{負数の表現}
  \begin{itembox}[l]{補数表現}
    \begin{itemize}
    \item $n$桁の$b$進数において \\ $b^n$から$x$を引いた数$y$を
      $x$に対する「$b$の\emph{補数}」と呼ぶ.
      \centerline{$y = b^n -x$ ~~~~  ($y$は$x$に対する$b$の補数)}
    \item $n$桁の$b$進数において \\$b^n-1$から$x$を引いた数$z$を
      $x$に対する「$(b-1)$の\emph{補数}」と呼ぶ.
      \centerline{$z = b^n - 1 -x$ ~~~~ ($z$は$x$に対する$(b-1)$の補数)}
    \end{itemize}
  \end{itembox}
\end{frame}

%==============================================================================
\begin{frame}
  \frametitle{負数の表現}
  \begin{itembox}[l]{2桁の10進数における補数の例}
    \begin{tabular}{ l c r c l }
      $b=10進数$    & ~ &       &   &              \\
      $n=2桁$       & ~ & $100$ &   &              \\
      $b^n = 100$   & ~ & $-25$ & ~ & 75は25に対す \\
      \cline{3-3}
      $x = 25$      &   &  $75$ &   & る10の補数   \\
                    &   &       &   &        \\
      $b=10進数$    & ~ &       &   &              \\
      $n=2桁$       & ~ &  $99$ &   &              \\
      $b^n-1 = 99$  & ~ & $-25$  & ~ & 74は25に対す \\
      \cline{3-3}
      $x = 25$      &   &  $74$ &   & る9の補数   \\
    \end{tabular}
  \end{itembox}
\end{frame}

%==============================================================================
\begin{frame}
  \frametitle{負数の表現}
  \begin{itembox}[l]{4桁の2進数における補数の例}
    \begin{tabular}{ l c r c l }
      $b=2進数$       & ~ &          &   & $0110_2$は  \\
      $n=4桁$         & ~ & $10000_2$ &   & $1010_2$に  \\
      $b^n = 10000_2$ & ~ & $-1010_2$ & ~ & 対する      \\
      \cline{3-3}
      $x = 1010_2$    &   & $0110_2$  &   & \underline{2の補数} \\
                      &   &           &   &            \\
      $b=2進数$       & ~ &           &   & $0101_2$は \\
      $n=4桁$         & ~ & $1111_2$  &   & $1010_2$に \\
      $b^n-1 = 1111_2$& ~ & $-1010_2$  & ~ & 対する      \\
      \cline{3-3}
      $x = 1010_2$    &   & $0101_2$  &   & \underline{1の補数} \\
    \end{tabular}
  \end{itembox}
\end{frame}

%==============================================================================
\begin{frame}
  \frametitle{負数の表現}
\begin{enumerate}
\item[(2)] \emph{1の補数による負数の表現}\\
  1の補数を負数の表現に使用する.
  \begin{itembox}[l]{4ビット2進数の1の補数($2^4 - 1 - x = z$)}
    \begin{tabular}{ c | r c r}
      \hline
      \hline
      もとの数(x) & \multicolumn{1}{|c}{補数へ変換}
      & & \multicolumn{1}{c}{補数(z)} \\
      \hline
      $0$  & $1111_2 - 0000_2$ & $=$ & $1111_2$ \\
      $1$  & $1111_2 - 0001_2$ & $=$ & $1110_2$ \\
      $2$  & $1111_2 - 0010_2$ & $=$ & $1101_2$ \\
      $3$  & $1111_2 - 0011_2$ & $=$ & $1100_2$ \\
      $4$  & $1111_2 - 0100_2$ & $=$ & $1011_2$ \\
      $5$  & $1111_2 - 0101_2$ & $=$ & $1010_2$ \\
      $6$  & $1111_2 - 0110_2$ & $=$ & $1001_2$ \\
      $7$  & $1111_2 - 0111_2$ & $=$ & $1000_2$ \\
    \end{tabular}
  \end{itembox}
\end{enumerate}
\end{frame}

%==============================================================================
\newcommand{\h}{$\vert$}
\begin{frame}
  \frametitle{負数の表現}
  \begin{itembox}[l]{1の補数を用いた符号付き数値}
    \small\begin{tabular}{ r r c c c c c c c c }
      $-7$  & $1000_2$ &--&--&--&--&--&--&--& +\\
      $-6$  & $1001_2$ &--&--&--&--&--&--&+ &\h\\
      $-5$  & $1010_2$ &--&--&--&--&--&+ &\h&\h\\
      $-4$  & $1011_2$ &--&--&--&--&+ &\h&\h&\h\\
      $-3$  & $1100_2$ &--&--&--&+ &\h&\h&\h&\h\\
      $-2$  & $1101_2$ &--&--&+ &\h&\h&\h&\h&\h\\
      $-1$  & $1110_2$ &--&+ &\h&\h&\h&\h&\h&\h\\
      $-0$  & $1111_2$ &+ &\h&\h&\h&\h&\h&\h&\h\\
      $+0$  & $0000_2$ &+ &\h&\h&\h&\h&\h&\h&\h\\
      $+1$  & $0001_2$ &--&+ &\h&\h&\h&\h&\h&\h\\
      $+2$  & $0010_2$ &--&--&+ &\h&\h&\h&\h&\h\\
      $+3$  & $0011_2$ &--&--&--&+ &\h&\h&\h&\h\\
      $+4$  & $0100_2$ &--&--&--&--&+ &\h&\h&\h\\
      $+5$  & $0101_2$ &--&--&--&--&--&+ &\h&\h\\
      $+6$  & $0110_2$ &--&--&--&--&--&--&+ &\h\\
      $+7$  & $0111_2$ &--&--&--&--&--&--&--& +\\
    \end{tabular}
  \end{itembox}
\end{frame}

%==============================================================================
\begin{frame}
  \frametitle{負数の表現}
  \begin{itemize}
  \item 1の補数の求め方 \\
    ビット反転\\
    ~~ $x = +3_{10} = 0011_2$ (もとの数)\\
    ~~ $y = -3_{10} = 1100_2$ (1の補数)\\
    \vfill
  \item 表現できる数値の範囲 \\
    4ビット: $-7 〜 +7$ ($-(2^3-1)〜+(2^3-1)$) \\
    nビット: $-(2^{n-1} - 1) 〜 +(2^{n-1} - 1)$
    \vfill
  \item 正負の判定 \\
    最上位ビットが \\
    ~~ 0:正の値を表現している.\\
    ~~ 1:負の値を表現している.
  \end{itemize}
\end{frame}

%==============================================================================
\begin{frame}
  \frametitle{負数の表現}
\begin{enumerate}
\item[(3)] \emph{2の補数による負数の表現}\\
  2の補数($2^n - x$)を負数の表現に使用する.
  \begin{itembox}[l]{4ビット2進数の2の補数($2^4 - x = y$)}
    \begin{tabular}{ c | r c r}
\hline
\hline
もとの数(x) & \multicolumn{1}{|c}{補数へ変換}
  & & \multicolumn{1}{c}{補数(y)} \\
\hline
$0$  & $1$\fbox{$0000$}$_2 - $\fbox{$0000$}$_2$ & $=$ & $1$\fbox{$0000$}$_2$ \\
$1$  & $1$\fbox{$0000$}$_2 - $\fbox{$0001$}$_2$ & $=$ & \fbox{$1111$}$_2$ \\
$2$  & $1$\fbox{$0000$}$_2 - $\fbox{$0010$}$_2$ & $=$ & \fbox{$1110$}$_2$ \\
$3$  & $1$\fbox{$0000$}$_2 - $\fbox{$0011$}$_2$ & $=$ & \fbox{$1101$}$_2$ \\
$4$  & $1$\fbox{$0000$}$_2 - $\fbox{$0100$}$_2$ & $=$ & \fbox{$1100$}$_2$ \\
$5$  & $1$\fbox{$0000$}$_2 - $\fbox{$0101$}$_2$ & $=$ & \fbox{$1011$}$_2$ \\
$6$  & $1$\fbox{$0000$}$_2 - $\fbox{$0110$}$_2$ & $=$ & \fbox{$1010$}$_2$ \\
$7$  & $1$\fbox{$0000$}$_2 - $\fbox{$0111$}$_2$ & $=$ & \fbox{$1001$}$_2$ \\
$8$  & $1$\fbox{$0000$}$_2 - $\fbox{$1000$}$_2$ & $=$ & \fbox{$1000$}$_2$ \\
    \end{tabular}
  \end{itembox}
\end{enumerate}
\end{frame}

%==============================================================================
\begin{frame}
  \frametitle{負数の表現}
  \begin{itembox}[l]{2の補数を用いた符号付き数値}
    \small\begin{tabular}{ r l c c c c c c c c }
    $-8$  & $1000_2$ &  &  &  &  &  &  &  &  \\
    $-7$  & $1001_2$ &--&--&--&--&--&--&--&+ \\
    $-6$  & $1010_2$ &--&--&--&--&--&--&+ &\h\\
    $-5$  & $1011_2$ &--&--&--&--&--&+ &\h&\h\\
    $-4$  & $1100_2$ &--&--&--&--&+ &\h&\h&\h\\
    $-3$  & $1101_2$ &--&--&--&+ &\h&\h&\h&\h\\
    $-2$  & $1110_2$ &--&--&+ &\h&\h&\h&\h&\h\\
    $-1$  & $1111_2$ &--&+ &\h&\h&\h&\h&\h&\h\\
    $ 0$  & $0000_2$ &+ &\h&\h&\h&\h&\h&\h&\h\\
    $ 1$  & $0001_2$ &--&+ &\h&\h&\h&\h&\h&\h\\
    $ 2$  & $0010_2$ &--&--&+ &\h&\h&\h&\h&\h\\
    $ 3$  & $0011_2$ &--&--&--&+ &\h&\h&\h&\h\\
    $ 4$  & $0100_2$ &--&--&--&--&+ &\h&\h&\h\\
    $ 5$  & $0101_2$ &--&--&--&--&--&+ &\h&\h\\
    $ 6$  & $0110_2$ &--&--&--&--&--&--&+ &\h\\
    $ 7$  & $0111_2$ &--&--&--&--&--&--&--&+ \\
    \end{tabular}
  \end{itembox}
\end{frame}

%==============================================================================
\begin{frame}
  \frametitle{負数の表現}
  \begin{itemize}
  \item 2の補数の求め方 \\
    ビット反転+1\\
    ~~ $x = +3_{10} = 0011_2$ (もとの数)\\
    ~~ $y = -3_{10} = 1100_2 + 1 = 1101_2$ (2の補数)\\
    \emph{元に戻すのもビット反転+1}\\
    ~~ $y = -3_{10} = 1101_2$ (2の補数)\\
    ~~ $y = +3_{10} = 0010_2 + 1 = 0011_2$ (もとの数)\\
    \vfill
  \item 表現できる数値の範囲 \\
    4ビット: $-8 〜 +7$ ($-2^3〜+(2^3-1)$) \\
    nビット: $-2^{n-1} 〜 +(2^{n-1} - 1)$
    \vfill
  \item 正負の判定 \\
    最上位ビットが \\
    ~~ 0:ゼロ,または,正の値を表現している.\\
    ~~ 1:負の値を表現している.
  \end{itemize}
\end{frame}

%==============================================================================
\begin{frame}
  \frametitle{負数の表現(問題1/2)}
\emph{問題9:}次の10進数を2の補数表現形式の4桁の2進数に変換しなさい.
\begin{enumerate}
\item[1)] $4_{10}$
\vfill
\item[2)] $-4_{10}$
\vfill
\item[3)] $5_{10}$
\vfill
\item[4)] $-5_{10}$
\vfill
\item[5)] $6_{10}$
\vfill
\item[6)] $-6_{10}$
\vfill
\end{enumerate}
\end{frame}

%==============================================================================
\begin{frame}
  \frametitle{負数の表現(問題2/2)}
\emph{問題10:}次の2の補数表現形式の4桁の2進数を10進数に変換しなさい.
\begin{enumerate}
\item[1)] $1001_2$
\vfill
\item[2)] $0111_2$
\vfill
\item[3)] $1101_2$
\vfill
\item[4)] $0011_2$
\vfill
\item[5)] $1011_2$
\vfill
\item[6)] $1100_2$
\vfill
\end{enumerate}
\end{frame}

\end{document}
