%\documentclass[dvipdfmx]{beamer}      % platex の場合
\documentclass[handout]{beamer}        % lualatex の場合
\usepackage{mySld}

\begin{document}
\title{基礎コンピュータ工学\\第3章 組み立て\\
       (パート3:ハンダ付け3)}
\date{}

\begin{frame}
  \titlepage
  \centerline{\url{https://github.com/tctsigemura/TecTextBook}}
  \vfill
  \centerline{本スライドの入手:
    \raisebox{-7mm}{\includegraphics[scale=0.3]{../Img/QRs3_3.png}}}
\end{frame}

%==============================================================================
%\begin{frame}
%  \frametitle
%  \tableofcontents
%\end{frame}

\section{組み立て}
%==============================================================================
\begin{frame}
  \frametitle{集合抵抗器とラダー抵抗器}
  \begin{minipage}{0.45\columnwidth}
    \includegraphics[scale=0.56]{../chap3/syuugou.pdf}
  \end{minipage}
  \begin{minipage}{0.54\columnwidth}
    \begin{center}
      {\footnotesize\begin{tabular}{l|l|l}
        \hline
        \hline
        \multicolumn{1}{c|}{記号} &
        \multicolumn{1}{c|}{型番} &
        \multicolumn{1}{c}{説明} \\
        \hline
        RA1,2 & M9-1-471   & 470Ω(8素子) \\
        &(L91S 471)&               \\
        RA3   & M9-1-391   & 390Ω(8素子) \\
        &(L91S 391)&               \\
        RA4   & M5-1-391   & 390Ω(4素子) \\
        &(L51S 391)&               \\
        RA5   & 8L103      & ラダー抵抗器  \\
      \end{tabular}}
    \end{center}
  \end{minipage}
\end{frame}

%==============================================================================
\begin{frame}
  \frametitle{LED(ランプ)}
  \vfill
  \centerline{\includegraphics[scale=0.65]{../Tikz/leds.pdf}\hspace{2cm}
    \includegraphics[scale=0.65]{../Tikz/leds2.pdf}}
  \vfill
  \begin{enumerate}
    \item[1.] 同じ色を一斉にに,アノード(+)だけハンダ付けする.
    \item[2.] LEDが垂直になっているか確認する.\\
      (垂直になっていない場合は,再度温めて修正する.)
    \item[3.] LEDが奥までささっているか確認する.
    \item[4.] カソードをハンダ付けする.
    \item[5.] リード線を切る.
  \end{enumerate}
  \vfill
\end{frame}

%==============================================================================
\begin{frame}
  \frametitle{スイッチ}
  \vfill
  \centerline{\includegraphics[scale=0.65]{../chap3/sw.pdf}\hspace{1cm}
    \includegraphics[scale=0.8]{../Tikz/sws.pdf}}
  \vfill
  \begin{enumerate}
  \item[1.] 足を穴にしっかり差し込む.
  \item[2.] 足のうち1本をハンダ付けする.
  \item[3.] 一列のスイッチについて1.,2.をする.
  \item[4.] スイッチが傾いていないか確認する.\\
    (傾いていた場合は,温め直して修正する.)
  \item[5.] 他の足をハンダ付けする.
  \end{enumerate}
  \vfill
\end{frame}

%==============================================================================
\begin{frame}
  \frametitle{入出力ポートコネクタ}
  \centerline{\includegraphics[scale=0.8]{../chap3/cnn.pdf}}
  \vfill
  \centerline{\small\begin{tabular}{l|l|l}
    \hline
    \hline
    \multicolumn{1}{c|}{記号} &
    \multicolumn{1}{c|}{型番} &
    \multicolumn{1}{c}{説明} \\
    \hline
    CN5 & なし & 大きい20ピンのコネクタ \\
  \end{tabular}}
  \vfill
  \begin{enumerate}
  \item[1.] 向きに注意!!
  \item[2.] 中央付近の一本をハンダ付けする.
  \item[3.] 向き,傾きを再度確認する.
  \item[4.] 残りの足をハンダ付けする.
  \end{enumerate}
  \vfill
\end{frame}

%==============================================================================
%\begin{frame}
%  \frametitle{}
%\end{frame}

\end{document}
