\renewcommand{\myincludegraphics}[2]{\includegraphics[#2]{appD/#1}}

\newpage
\onecolumn
\chapter{参考資料}

TeC7をもっと活用してもらうために,
TeC7の基板回路図とFPGAのピン配置表を掲載します.
TeC7の全設計データは,
\url{https://github.com/tctsigemura/TeC7}
に公開してあります.
また,以下に掲載する回路図やピン配置表の電子データは,
\url{https://github.com/tctsigemura/TeC7/blob/master/Doc/PCB}
に公開してあります.

\section{TeC7c基板回路図}
\figref{appD:kairoC}にTeC7c本体のプリント基板回路図を示します.

\section{TeC7cピン配置表}
\figref{appD:pinC}にTeC7cに搭載されている
Xilinx Spartan-6 FPGA (XC6LX9-2TQG144C)の
ピンがどのように利用されているかを示します.

\section{TeC7d基板回路図}
\figref{appD:kairoD}にTeC7d本体のプリント基板回路図を示します.

\section{TeC7dピン配置表}
\figref{appD:pinD}にTeC7dに搭載されている
Xilinx Spartan-6 FPGA (XC6LX9-2TQG144C)の
ピンがどのように利用されているかを示します.

\newpage
\myfigureNA{btph}{angle=90,scale=0.81}{TeC7c-v-crop.pdf}
           {TeC7c基板回路図}{appD:kairoC}

\newpage
\myfigureNA{btph}{angle=90,scale=0.86}{TeC7cPin-crop.pdf}
           {TeC7cピン配置}{appD:pinC}

\newpage
\myfigureNA{btph}{angle=90,scale=0.81}{TeC7d-v-crop.pdf}
           {TeC7d基板回路図}{appD:kairoD}

\newpage
\myfigureNA{btph}{angle=90,scale=0.86}{TeC7dPin-crop.pdf}
           {TeC7dピン配置}{appD:pinD}
